%%%%%%%%%%%%%%%%%%%%%%%%%%%%%%%%%%%%%%%%%%%%%%%%%%%%%%%%%%%%%%%%%%%%%%%%%%%%%%%
%% Descr:       Vorlage für Berichte der DHBW-Karlsruhe
%% Author:      Prof. Dr. Jürgen Vollmer, vollmer@dhbw-karlsruhe.de
%% Modified :	Kristin Agne, TINF19B4
%% -*- coding: utf-8 -*-
%%%%%%%%%%%%%%%%%%%%%%%%%%%%%%%%%%%%%%%%%%%%%%%%%%%%%%%%%%%%%%%%%%%%%%%%%%%%%%%

\chapter{Programming Principles}

\section{SOLID}
SOLID steht für fünf Prinzipien, die im Folgenden betrachtet werden.

    \subsection{Single Responsibility Principle}
    
        \subsubsection{Analyse}
        Das Prinzip wird bei den RepositoryInterface Klassen eingesetzt. Hierbei hat jedes RepositoryInterface nur eine genau definierte Funktion. Diese Funktion ist die Anbindung an die Datenbank.
        
        \subsubsection{Begründung}
        Der Grund dafür ist, dass eine Klasse, Methode, etc. nur eine Zuständigkeit haben soll. Es kapselt so die Zuständigkeit, diese sind klar definiert, da jede Klasse etc. nur einen Nutzen hat. Das führt dazu, dass Änderungen der Software nur zu Änderungen in einem eingegrenzten Bereich führen.

    \subsection{Open Closed Principle}
    
        \subsubsection{Analyse}
        Hierbei geht es darum, dass Klassen einfach erweiterbar sind, ohne sie grundsätzlich zu ändern. In dem Projekt kommt dieses Prinzip zum Beispiel in den Application Service Klassen oder den Controller Klassen zum Einsatz. Hierbei kann der Controller um einen Endpunkt erweitert werden, ohne dass die Klasse grundlegend modifiziert werden muss. Auch bei den Service Klassen trifft dies zu.
        
        \subsubsection{Begründung}
        Der Grund für die Benutzung dieses Prinzips ist, dass eine Erweiterung an Methoden oft vorkommt. Wenn man jedes Mal die Klasse von Grund auf ändern müsste, wäre das sehr aufwendig. Durch das Open Closed Principle kann man neue Methoden ohne großen Aufwand hinzufügen.

    \subsection{Liskov Substitution Principle}
        Unter dem Liskov Substitution Principle wird zusammengefasst, dass eine Instanz einer Klasse durch eine Instanz einer anderen Klasse, die entweder von dieser Klasse erbt oder das gleiche Interface erweitert, ersetzt werden kann. Hierbei darf es nicht zu einem logischen Bruch innerhalb der Anwendung kommen. In diesem Projekt gibt es zwar keine Vererbung, aber es wurden Interfaces benutzt. Dieses Prinzip kommt also bei den Repository Interfaces zum Einsatz. Die Bridge Klassen (z.B. GenreBridge) implementieren hierbei das Repository Interface und jede Instanz der Bridge Klasse kann durch eine Instanz ersetzt werden, die das gleiche Interface implementiert. Dies wird mit dem Liskov Substitution Principle umgesetzt, um die implementierte Logik in der Bridge Klasse einfach austauschen zu können, ohne alle Klassen anpassen zu müssen, die von den Repository Interfaces abhängig sind (wie die Application Services).

    \subsection{Interface Segregation Principle}
    
        \subsubsection{Analyse}
        Das Interface Segregation Principle wurde in den Repository Interfaces verwendet. Hierbei wurden keine unnötigen Schnittstellenteile implementiert.
        
        \subsubsection{Begründung}
        Der Grund hierfür ist, dass Schnittstellen so kleiner sind und nur verwendete Methoden zwingend implementiert werden müssen. Dadurch wird die Wartbarkeit erheblich verbessert und der Code wird kompakter und besser wiederverwendbar. Die implementierten Klassen gewinnen so an Übersichtlichkeit, fachlich zusammengehöriges Verhalten wird so gruppiert und die Modularität wird dadurch verbessert.

\subsection{Dependency Inversion Principle}

        \subsubsection{Analyse}
        Das Dependency Inversion Principle kommt vor allem bei objektorientierten Entwürfen zum Einsatz und besagt, dass höhere Module nicht von niedrigeren Modulen abhängen sollen. Die einzelnen Module stellen hierbei verschiedenen Ebenen dar. Die Ebenen haben verschiedene Ordnungen, und sind abhängig voneinander. In diesem Projekt ist die niedrigste Ebene die Plugin-Schicht (siehe \cref{cleanArchitecture}). Module niedriger Ebenen sind immer von Modulen höherer Ebenen abhängig, da Module niedrigerer Ebenen die Vorgänge der höheren Ebene spezifizieren.
        
        \subsubsection{Begründung}
        Der Grund, warum man die verschiedenen Ebenen in Module aufteilt und somit eine Abhängigkeit von niedrigen zu höheren Ebenen realisiert, ist, dass man so dem Prinzip der Hierarchie nicht widersprechen kann. Des Weiteren wird so die Vorstellung der Clean Architecture zwangsweise umgesetzt und Veränderungen in der Architektur oder des Design sind leicht umsetzbar. So wird auch die Komplexität des gesamten Projekts verringert.

\section{GRASP}
GRASP steht für General Responsibility Assignment Software Patterns. Darunter werden neun Muster verstanden, welche im Folgenden näher betrachtet werden.

	\subsection{Low Coupling and High Cohesion (Service und Controller)}
	Im Folgenden wird die Betrachtung von Low Coupling und High Cohesion in den Services und Controllern betrachtet.
		\subsubsection{Low Coupling}
			\myparagraph{Analyse}
			Bei dem Prinzip Low Coupling geht es um die Minimierung der Abhängigkeiten einer Klasse von der Umgebung. Hierbei sollten die einzelnen Klassen und Objekte möglichst wenig untereinander vernetzt sein, sodass die Abhängigkeiten so gering wie möglich gehalten sind. Dieses Prinzip wurde in einigen ApplicationServices umgesetzt, aber nicht in allen. Der ActorApplicationService oder der GenreApplicationService sind jeweils nur von einem Repository, von dem ApplicationRepository bzw. dem GenreRepository abhängig. Hier ist das Prinzip des Low Coupling umgestezt. Im Gegensatz dazu wurde das Prinzip im EpisodeApplicationService nicht umgesetzt, da hier eine Abhängigkeit zu vier Repositories besteht. Hier spricht man von High Coupling.
			
			\myparagraph{Begründung}
			Der Grund für das Einsetzen von Low Coupling ist die leichte Anpassbarkeit. Durch die geringen Abhängikeiten können ActorApplicationService und GenreApllication im Gegensatz zum EpisodeApllicationService leicht angepasst werden. Auch die Verständlichkeit der beiden Klassen mit Low Coupling ist einfacher und man kann die beiden Klassen besser wieder verwenden. Zusätzlich ist das Testen von Klassen mit Low Coupling einfacher, da man nur auf wenige Abhängigkeiten achten muss.
		
		\subsubsection{High Cohesion}
			\myparagraph{Analyse}
			Bei dem Prinzip High Cohesion geht es um die Vermeidung von verschiedenen Verantwortlichkeiten bzw. Aufgaben innerhalb einer Klasse. Hierbei wird betrachtet, inwieweit die Objekte und Attribute einer Klasse zusammenarbeiten und wie viel sie über andere Objekte wissen müssen. Im Projekt wurde dieses Prinzip zum Beispiel in den Controllern umgesetzt. Hierbei bekommen die Controller eine Anzahl von Parametern und eine Methode, und sie verarbeiten diese Daten mit Hilfe des Services. Dabei haben die Controller nur eine Art von Verantwortlichkeit und somit ist das Prinzip der hohen Kohäsion hier umgesetzt.
			
			\myparagraph{Begründung}
			Der Grund für die Umsetzung des Prinzips ist, dass man so die Übersichtlichkeit der Klasse deutlich verbessert und die Komplexität minimiert.
			
		\subsubsection{Zusammenhang Low Coupling und High Cohesion}
		Diese zwei Prinzipien stehen bei den Service und Controller Klassen in negativer Korrelation zueinander. Je mehr Verantwortlichkeiten man in andere Klassen auslagert, desto höher wird die Cohesion zwischen den Klassen und desto höher wird auch die Abhängigkeit. High Cohesion geht also mit High Coupling einher, Low Coupling und High Cohesion voll umzusetzen ist schwierig, man sollte ein Mittel finden, bei dem das Coupling möglichst low und die Cohasion möglichst high ist.
	
	\subsection{Loose Coupling und High Cohesion (Domain)}
	Im Bereich der Domäne verhält sich die Verbindung zwischen Low Coupling und High Cohesion synchron.
	
		\subsubsection{Loose Coupling}
		
		\myparagraph{Analyse}
		Bei dem Prinzip Loose Coupling geht es um die Minimierung der Abhängigkeiten einer Klasse von der Umgebung. Betrachtet man die Klassen innerhalb der Domäne, sieht man, dass hier Tight Coupling ist. Da die beispielsweise die Klasse Genre eine List<Serie> und jede Serie ein Genre Objekt hat, sind die beiden Klassen gekoppelt. 
		
		\myparagraph{Begründung}
		Der Grund für diese Umsetzung und die hohe Kopplung ist, wie in Spring die Entities angelegt wurden und wie mit deren Beziehungen umgegangen wird. Hierbei werden komplette Objekte anstelle von ihren Id's referenziert.
		
		\subsubsection{High Cohesion}
		
		\myparagraph{Analyse}
		Bei dem Prinzip High Cohesion geht es um die Vermeidung von verschiedenen Verantwortlichkeiten bzw. Aufgaben innerhalb einer Klasse. Innerhalb der Domäne hätte man aufgrund von Tight Coupling eigentlich eine Low Cohesion, da durch die hohe Abhängigkeit von der Genre und Serie Klasse Funktionen gegenseitig übernommen werden müssten. Jedoch ist dies durch Spring nicht der Fall, weil das Genre theoretisch nur in der Serie gespeichert ist und beim Persistieren die List<Serie> in der Genre Klasse von Spring automatisch geupdatet wird.
		
		\myparagraph{Begründung}
		Auch hierfür ist der Grund Verbesserung der Übersichtlichkeit der Klasse sowie die Minimierung der Komplexität.
		
		\subsubsection{Zusammenhang Loose Coupling und High Cohesion}
		Normalerweise stehen Loose Coupling und High Cohesion in der Domäne synchron in Beziehung, da bei einer geringen Anzahl von Abhängikeit auch eine hohe Kohäsion umgesetzt wird. Allerdings ist durch Spring und die Verwendung von Hibernate hier Tight Coupling umgesetzt, und trotzdem ist noch eine hohe Kohäsion vorhanden.

    \subsection{Creator}
    Unter dem Erzeuger-Prinzip versteht man das Prinzip, dass eine Instanz einer Klasse nur von einem Experten A erzeugt wird. In diesem Projekt findet das Creator-Pattern den Einsatz beispielweise im RatingApplicationService. Dieser sammelt die benötigten Initialisierungsdaten und instanziiert dann ein RatingAverage. Man könnte sagen, der RatingApplicationService ist der Experte A für die Erzeugung eines Rating Average.
    
    \subsection{Controller}
    Das Controller-Pattern findet in dem Projekt in den Controllern in der Plugins-Schicht Anwendung. Sie sind dafür zuständig, um die Benutzereingaben zu empfangen und zu verarbeiten. Ein Request wird mit Parametern von einem Controller entgegengenommen und von diesem wird die entsprechende Methode aufgerufen.
    
    \subsection{Indirection}
    Das Indirection Pattern wird verwendet, um Low Coupling umzusetzen. Es wird ein sogenannter Vermittler zwischen Client und Server eingesetzt. In diesem Projekt findet dieses Pattern keinen Einsatz, da die Leistungsfähigkeit so vermindert wird.
    
    \subsection{Information Expert}
    Bei dem Information Expert Pattern geht es darum, dass eine Klasse nur genau die Informationen kennt, die sie auch benötigt, um ihrer Verantwortung gerecht zu werden. Das heißt, eine Aufgabe wird immer von der Klasse übernommen, die das meiste benötigte Wissen besitzt. Dieses Pattern fördert high cohesion. Umgesetzt wird dieses Pattern in den Entitäten und Repositories. Die Entitäten kapseln das Wissen, sodass nur mit Gettern und Settern darauf zugegriffen werden kann. Bei den Repositories übernimmt immer genau das Repository die Aufgabe, welches das meiste Wissen hat.
    
    \subsection{Polymorphism}
    Das Polymorphism-Pattern wird in diesem Projekt nicht umgesetzt, da es keine Anwendungsfälle (beispielsweise Vererbung oder verschiedene Ausprägungen einer Interface Methode) in diesem Projekt gibt.
    
    \subsection{Protected Variations}
    Das Protected Variations Pattern wird in diesem Projekt durch die Repository Interfaces in der Domänenschicht, die in der Plugins Schicht implementiert werden, umgesetzt. Dadurch wird die eigentliche Implementierung in der Plugins Schicht über die Interfaces verschleiert. So hat eine Veränderung an der Implementierung keine Auswirkung auf das restliche System.
    
    \subsection{Pure Fabrication}
    Das Pure Fabrication Pattern stellt Klassen dar, die in der Domäne nicht existieren. In diesen Klassen werden beispielsweise Methoden ausgelagert und die Klassen werden als Erfindung bezeichnet, da sie nichts Reales aus der Domänenschicht repräsentieren. Dieses Pattern wird zur Unterstützung von high cohesion und low coupling eingesetzt. In diesem Projekt wird Pure Fabrication durch die Data Transfer Objects (DTO) in der Adapter Schicht umgesetzt. Durch diese Klassen wird beispielsweise das Frontend von der Domäne entkoppelt. Eine Anpassung der Domäne führt nicht zwingend zu einer Anpassung des Frontends und umgekehrt.

\section{DRY}
Das Prinzip DRY bedeutet \hk{Don't Repeat yourself}. Hierbei handelt es sich um ein Prinzip des Clean Code, damit Codeabschnitte nicht wiederholt, sondern stattdessen ausgelagert und wiederverwendet werden.

    \subsection{Analyse}
    Diese Prinzip ist ein grundlegendes Prinzip und wurde prinzipiell in dem gesamten Projekt umgesetzt. Im \cref{refactoring} sieht man die Extraktion von Teilen einer Methode in einzelne neue Methoden. Diese neuen Methoden werden anschließend auch von anderen Methoden benutzt, wodurch eine Wiederholung vermieden wird.
    
    \subsection{Begründung}
    Diese Prinzip hat mehrere Vorteile, da man so weniger Code schreiben muss und sich somit Zeit spart. Außerdem nimmt die Übersichtlichkeit enorm zu, wenn man gleiche Teile in Methoden auslagert und mehrfach verwendet. Nicht nur die Übersichtlichkeit, sondern auch die Verständlichkeit und Fehleranfälligkeit des Codes wird so verbessert. Des Weiteren wird die Wartbarkeit vereinfacht, weil man so bei Logikfehlern oder Veränderungen nur eine Methode und nicht alle Codestellen anpassen muss.